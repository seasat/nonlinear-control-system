\documentclass[a4paper, listof=totoc]{scrreport}
\usepackage{attitude_control}

\newglossaryentry{sy:dcm}{
    name = \ensuremath{\mat{C}},
    description = {\gls{ac:dcm}},
    sort = C,
    type = symbols,
}

\newglossaryentry{sy:c1}{
    name = \ensuremath{\mat{C_1}},
    description = {Single axis rotation matrix for roll},
    sort = C1,
    type = symbols,
}

\newglossaryentry{sy:c2}{
    name = \ensuremath{\mat{C_2}},
    description = {Single axis rotation matrix for pitch},
    sort = C2,
    type = symbols,
}

\newglossaryentry{sy:c3}{
    name = \ensuremath{\mat{C_3}},
    description = {Single axis rotation matrix for yaw},
    sort = C3,
    type = symbols,
}

\newglossaryentry{sy:roll}{
    name = \ensuremath{\theta_1},
    description = {Roll angle},
    sort = theta1,
    type = symbols,
}

\newglossaryentry{sy:pitch}{
    name = \ensuremath{\theta_2},
    description = {Pitch angle},
    sort = theta2,
    type = symbols,
}

\newglossaryentry{sy:yaw}{
    name = \ensuremath{\theta_3},
    description = {Pitch angle},
    sort = theta3,
    type = symbols,
}


\subject{Attitude Dynamics and Control (AE4313)}
\title{Nonlinear Spacecraft Attitude Control System Design}
\subtitle{Practical Exercise Project Assignment 2}
\author{Lars Blümler\\6149065}
\publishers{\includegraphics{tu_delft_logo.pdf}}


\begin{document}
% front matter
\maketitle
\pagenumbering{roman}
\tableofcontents
\listoffigures
\begingroup
    \let\clearpage\relax
    \listoftables
\endgroup
\clearpage


% main matter
\pagenumbering{arabic}
\chapter{Methodology}
\section{Coordinate Systems}
\subsection{Inertial Frame}
\subsection{\texorpdfstring{\acrlong{lvlh}}{Local Vertical Local Horizontal} Frame}
\subsection{Body Frame}


\section{Kinematics}
\subsection{\texorpdfstring{\acrlong{ac:dcm}}{Direction Cosine Matrix}}
The \glsfull{sy:dcm} is a matrix that describes the linear transformation between two coordinate frames.
In three dimension, $\gls{sy:dcm} \in \mathbb{R}^{3 \times 3}$ and the definition is given by \cref{eqt:dcm}.
\begin{equation} \label{eqt:dcm}
    \gls{sy:dcm} = \begin{bmatrix}
        \gls{sy:dcm}_{11} & \gls{sy:dcm}_{12} & \gls{sy:dcm}_{13} \\
        \gls{sy:dcm}_{21} & \gls{sy:dcm}_{22} & \gls{sy:dcm}_{23} \\
        \gls{sy:dcm}_{31} & \gls{sy:dcm}_{32} & \gls{sy:dcm}_{33}
    \end{bmatrix}
\end{equation}

A transformation matrix $\gls{sy:dcm}^{B/A}$ from frame $A$ to frame $B$ is constructed from the basis vectors of frame $A$ ($\gls{sy:a}_1$, $\gls{sy:a}_2$, $\gls{sy:a}_3$) as expressed in frame $B$ ($\gls{sy:b}_1$, $\gls{sy:b}_2$, $\gls{sy:b}_3$).
In the case of both bases consisting of unit vectors, the \gls{ac:dcm} elements become equal to the cosine of the angle between each pair of basis vectors (see \cref{eqt:dcm_basis}), hence the name cosine matrix.
\begin{equation} \label{eqt:dcm_basis}
    \gls{sy:dcm}^{B/A} = \begin{bmatrix}
        \gls{sy:b}_1 \cdot \gls{sy:a}_1 & \gls{sy:b}_1 \cdot \gls{sy:a}_2 & \gls{sy:b}_1 \cdot \gls{sy:a}_3 \\
        \gls{sy:b}_2 \cdot \gls{sy:a}_1 & \gls{sy:b}_2 \cdot \gls{sy:a}_2 & \gls{sy:b}_2 \cdot \gls{sy:a}_3 \\
        \gls{sy:b}_3 \cdot \gls{sy:a}_1 & \gls{sy:b}_3 \cdot \gls{sy:a}_2 & \gls{sy:b}_3 \cdot \gls{sy:a}_3
    \end{bmatrix}
    =
    \begin{bmatrix}
        \cos \theta_{11} & \cos \theta_{12} & \cos \theta_{13} \\
        \cos \theta_{21} & \cos \theta_{22} & \cos \theta_{23} \\
        \cos \theta_{31} & \cos \theta_{32} & \cos \theta_{33}
    \end{bmatrix}
\end{equation}


\subsection{Euler Angles}
Euler angles are a set of three angles that represent an orientation, rotation or attitude as three successive, single axis rotations.
The rotations can be performed about any axis in any order, so long as no two rotations are about the same axis.
For any single axis rotation about axis $i$ with angle $\theta_i$, the rotation axis remains unchanged leaving $\gls{sy:a}_i = \gls{sy:b}_i$.
The resulting \glspl{ac:dcm} are given by \cref{eqt:c1,eqt:c2,eqt:c3}.

\parbox{.49\textwidth}{
\begin{equation} \label{eqt:c1}
    \gls{sy:c1}(\gls{sy:roll}) = \begin{bmatrix}
        1 & 0 & 0 \\
        0 & \cos(\gls{sy:roll}) & \sin(\gls{sy:roll}) \\
        0 & -\sin(\gls{sy:roll}) & \cos(\gls{sy:roll})
    \end{bmatrix}
\end{equation}
}
\parbox{.49\textwidth}{
\begin{equation} \label{eqt:c2}
    \gls{sy:c2}(\gls{sy:pitch}) = \begin{bmatrix}
        \cos(\gls{sy:pitch}) & 0 & -\sin(\gls{sy:pitch}) \\
        0 & 1 & 0 \\
        \sin(\gls{sy:pitch}) & 0 & \cos(\gls{sy:pitch})
    \end{bmatrix}
\end{equation}
}
\begin{equation} \label{eqt:c3}
    \gls{sy:c3}(\gls{sy:yaw}) = \begin{bmatrix}
        \cos(\gls{sy:yaw}) & \sin(\gls{sy:yaw}) & 0 \\
        -\sin(\gls{sy:yaw}) & \cos(\gls{sy:yaw}) & 0 \\
        0 & 0 & 1
    \end{bmatrix}
\end{equation}

Using a $3 \rightarrow 2 \rightarrow 1$ rotation sequence using the \glsfull{sy:yaw}, \glsfull{sy:pitch} and \glsfull{sy:roll}, the resulting \gls{ac:dcm} is given by \cref{eqt:dcm_euler}.
\begin{equation} \label{eqt:dcm_euler}
    \gls{sy:dcm}^{B/A} = \gls{sy:c1}(\gls{sy:roll}) \, \gls{sy:c2}(\gls{sy:pitch}) \, \gls{sy:c3}(\gls{sy:yaw})
\end{equation}


\subsection{Quaternions}
\subsection{Angular Rates}


\section{Controller Design}
\subsection{Linearization}
\subsection{\texorpdfstring{\acrlong{pd}}{Proportional Differential} Controller}
\subsection{\texorpdfstring{\acrlong{ndi}}{Nonlinear Dynamic Inversion}}
\subsection{\texorpdfstring{\acrlong{tss}}{Time Scale Separation}}
\subsection{\texorpdfstring{\acrlong{indi}}{Incremental Dynamic Inversion}}


\chapter{Results}
\chapter{Discussion}


% back matter
\printglossary[type=\acronymtype]
\begingroup
    \let\clearpage\relax
    \printglossary[type=symbols, style=long3col]
\endgroup
\printbibliography

\end{document}
