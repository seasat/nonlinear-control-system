
\section{Controller Design}
\subsection{State Space}
The system can be represented in state space form, where the state vector depends on the attitude representation used.
The control problem is a second order system, where the controller acts on the torque which is in turn integrated twice to obtain the attitude, which is the control variable.
As such both the attitude and the angular velocity form part of the state vector.

\parbox{.49\linewidth}{
\begin{equation}
    \gls{sy:x_dot} =
    \begin{bmatrix}
        \gls{sy:ypr_dot} \\
        \gls{sy:omega_dot}
    \end{bmatrix}
\end{equation}
}
\parbox{.49\linewidth}{
\begin{equation}
    \gls{sy:x_dot} =
    \begin{bmatrix}
        \gls{sy:q_dot} \\
        \gls{sy:omega_dot}
    \end{bmatrix}
\end{equation}
}

\gls{sy:omega_dot} is defined by the dynamics of the system and calculated through ???.
\gls{sy:ypr_dot} and \gls{sy:q_dot} arise from the kinematics of the system and are calculated through \cref{eqt:omega_to_theta_dot,eqt:omega_to_q_dot}.
Both cases are highly nonlinear systems, to which different control strategies are subsequently applied.


\subsection{Linearization}
\subsection{\texorpdfstring{\acrlong{pd}}{Proportional Differential} Controller}
\subsection{\texorpdfstring{\acrlong{ndi}}{Nonlinear Dynamic Inversion}}
\subsection{\texorpdfstring{\acrlong{tss}}{Time Scale Separation}}
\subsection{\texorpdfstring{\acrlong{indi}}{Incremental Dynamic Inversion}}