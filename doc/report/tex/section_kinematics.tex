\section{Kinematics}
\subsection{Orbit Kinematics}
Treated in this exercise is a spacecraft in a circular orbit of \glsdesc{sy:h} \qty{700}{\km} around Earth.
The orbital frequency or \glsdesc{sy:mean_motion} is calculated using \cref{eqt:mean_motion}.
\begin{equation} \label{eqt:mean_motion}
    \gls{sy:mean_motion} = \sqrt{\frac{\gls{sy:mu_earth}}{\gls{sy:r}^3}} = \sqrt{\frac{\gls{sy:mu_earth}}{(\gls{sy:r}_E + h)^3}}
\end{equation}

Using $\gls{sy:mu_earth} = \qty{3.98600442e14}{\meter^3\per\second^2}$ and $\gls{sy:r}_E = \qty{6.378137e6}{\m}$, the mean motion is calculated to be $\gls{sy:mean_motion} = \qty{1.060206e-3}{\per\second}$.

As such, the rotation rate of the \gls{lvlh} frame with respect ot the inertial frame is $\gls{sy:omega} = \gls{sy:mean_motion} \gls{sy:c2}$.
Depending on the attitude representation used, this results in \cref{eqt:omega_lvlh_ypr, eqt:omega_lvlh_q}.

\parbox{.5\textwidth}{
\begin{equation} \label{eqt:omega_lvlh_ypr}
    \vec{\omega}_{\gls{lvlh}} = \begin{bmatrix}
        0 \\
        0 \\
        \gls{sy:omega}
    \end{bmatrix}
\end{equation}
}
\parbox{.5\textwidth}{
\begin{equation} \label{eqt:omega_lvlh_q}
    \vec{\omega}_{\gls{lvlh}} = \begin{bmatrix}
        0 \\
        0 \\
        \gls{sy:omega} \\
        0
    \end{bmatrix}
\end{equation}
}
