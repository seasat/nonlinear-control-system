\section{Kinematics}
\subsection{Orbit Kinematics}
Treated in this exercise is a spacecraft in a circular orbit of \glsdesc{sy:h} \qty{700}{\km} around Earth.
The orbital frequency or \glsdesc{sy:mean_motion} is calculated using \cref{eqt:mean_motion}.
\begin{equation} \label{eqt:mean_motion}
    \gls{sy:mean_motion} = \sqrt{\frac{\gls{sy:mu_earth}}{\gls{sy:r}^3}} = \sqrt{\frac{\gls{sy:mu_earth}}{(\gls{sy:r}_E + h)^3}}
\end{equation}

Using $\gls{sy:mu_earth} = \qty{3.98600442e14}{\meter^3\per\second^2}$ and $\gls{sy:r}_E = \qty{6.378137e6}{\m}$, the mean motion is calculated to be $\gls{sy:mean_motion} = \qty{1.060206e-3}{\per\second}$.

As such, the rotation rate of the \gls{lvlh} frame with respect ot the inertial frame is $\gls{sy:omega} = \gls{sy:mean_motion} \gls{sy:c2}$.
Depending on the attitude representation used, this results in \cref{eqt:omega_lvlh}.
\begin{equation} \label{eqt:omega_lvlh}
    \gls{sy:omega}_{\gls{lvlh}} = n \begin{bmatrix}
        \cos{\gls{sy:roll}} \sin{\gls{sy:yaw}} \\
        \sin{\gls{sy:roll}} \sin{\gls{sy:pitch}} \sin{\gls{sy:yaw}} + \cos{\gls{sy:roll}} \cos{\gls{sy:yaw}} \\
        \cos{\gls{sy:roll}} \sin{\gls{sy:pitch}} \sin{\gls{sy:yaw}} - \sin{\gls{sy:roll}} \cos{\gls{sy:yaw}}
    \end{bmatrix}
    =
    n \begin{bmatrix}
        2 (q_1 q_2 + q_3 q_4) \\
        1 - 2 (q_1^2 + q_3^2) \\
        2 (q_1 q_3 - q_1 q_4)
    \end{bmatrix}
\end{equation}


\subsection{Yaw Pitch Roll Rates}
\begin{multline} \label{eqt:theta_dot_to_omega_inertial}
    \gls{sy:omega}^I = \begin{bmatrix} \omega_1 \\ \omega_2 \\ \omega_3 \end{bmatrix}
    =
    \begin{bmatrix} \gls{sy:roll_dot} \\ 0 \\ 0 \\ \end{bmatrix}
    + \gls{sy:c1}(\gls{sy:roll})
    \begin{bmatrix} 0 \\ \gls{sy:pitch_dot} \\ 0 \\ \end{bmatrix}
    + \gls{sy:c1}{\gls{sy:roll}} \gls{sy:c2}(\gls{sy:pitch})
    \begin{bmatrix} 0 \\ 0 \\ \gls{sy:yaw_dot} \end{bmatrix}
    \\
    =
    \begin{bmatrix}
        1 & 0 & -\sin{\gls{sy:pitch}} \\
        0 & \cos{\gls{sy:roll}} & \sin{\gls{sy:roll}} \cos{\gls{sy:pitch}} \\
        0 & -\sin{\gls{sy:roll}} & \cos{\gls{sy:roll}} \cos{\gls{sy:pitch}}
    \end{bmatrix}
    \begin{bmatrix}
        \gls{sy:roll_dot} \\
        \gls{sy:pitch_dot} \\
        \gls{sy:yaw_dot}
    \end{bmatrix}
\end{multline}
\begin{equation} \label{eqt:omega_to_theta_dot}
    \gls{sy:ypr_dot}
    =
    \begin{bmatrix}
        \gls{sy:roll_dot} \\
        \gls{sy:pitch_dot} \\
        \gls{sy:yaw_dot}
    \end{bmatrix}
    =
    \frac{1}{\cos{\gls{sy:pitch}}}
    \begin{bmatrix}
        \cos{\gls{sy:pitch}} & \sin{\gls{sy:roll}} \sin{\gls{sy:pitch}} & \cos{\gls{sy:roll}} \sin{\gls{sy:pitch}} \\ 
        0 & \cos{\gls{sy:roll}} \cos{\gls{sy:pitch}} & -\sin{\gls{sy:roll}} \cos{\gls{sy:pitch}} \\
        0 & \sin{\gls{sy:roll}} & \cos{\gls{sy:roll}}
    \end{bmatrix}
    + \frac{\gls{sy:mean_motion}}{\cos{\gls{sy:pitch}}}
    \begin{bmatrix}
        \sin{\gls{sy:yaw}} \\
        \cos{\gls{sy:pitch}} \sin{\gls{sy:yaw}} \\
        \sin{\gls{sy:pitch}} \sin{\gls{sy:yaw}}
    \end{bmatrix}
\end{equation}


\subsection{Quaternion Rates}
\begin{equation} \label{eqt:omega_to_q_dot}
    \gls{sy:q_dot} = \begin{bmatrix}
        \dot{q}_1 \\
        \dot{q}_2 \\
        \dot{q}_3 \\
        \dot{q}_4
    \end{bmatrix}
    =
    \frac{1}{2} \begin{bmatrix}
        q_4 & -q_3 & q_2 & q_1 \\
        q_3 & q_4 & -q_1 & q_2 \\
        -q_2 & q_1 & q_4 & q_3 \\
        -q_1 & -q_2 & -q_3 & q_4
    \end{bmatrix}
    \left(
        \begin{bmatrix}
            \omega_1 \\
            \omega_2 \\
            \omega_3 \\
            0
        \end{bmatrix}
        - n \begin{bmatrix}
            2 (q_1 q_2 + q_3 q_4) \\
            1 - 2 (q_1^2 + q_3^2) \\
            2 (q_1 q_3 - q_1 q_4) \\
            0
        \end{bmatrix}
    \right)
\end{equation}

The use of the extended body angular rates vector as well as the extended \gls{lvlh} rotation rate vector allows for square matrix that can be used to invert the relationship.
Body axes angular rates can then be calculated following \cref{eqt:omega_to_q_dot}.
\begin{equation}
    \begin{bmatrix}
        \omega_1 \\
        \omega_2 \\
        \omega_3 \\
        0
    \end{bmatrix}
    =
    2 \begin{bmatrix}
        q_4 & q_3 & -q_2 & -q_1 \\
        -q_3 & q_4 & q_1 & -q_2 \\
        q_2 & -q_1 & q_4 & -q_3 \\
        q_1 & q_2 & q_3 & q_4
    \end{bmatrix}
    \gls{sy:q_dot}
    - n \begin{bmatrix}
        2 (q_1 q_2 + q_3 q_4) \\
        1 - 2 (q_1^2 + q_3^2) \\
        2 (q_1 q_3 - q_1 q_4) \\
        0
    \end{bmatrix}
\end{equation}
